% Presentation which is used in an educational video. The presentation is
% recorded in a screencast.
%
% Written in 2017 by Stephan Kulla
%
% To the extent possible under law, the author(s) have dedicated all copyright
% and related and neighboring rights to this software to the public domain
% worldwide. This software is distributed without any warranty.
%
% You should have received a copy of the CC0 Public Domain Dedication along
% with this software. If not, see
% <http://creativecommons.org/publicdomain/zero/1.0/>.

\documentclass[aspectratio=169,xcolor={dvipsnames,svgnames,table}]{beamer}

\usepackage[utf8]{inputenc}
\usepackage{centernot}
\beamertemplatenavigationsymbolsempty

\usetheme{metropolis}
\usecolortheme{owl}

\setbeamertemplate{footline}{}

\title{Unstetigkeitsbeweise mit dem Folgenkriterium (Beweisschema)}
\date{}

\newcommand*{\N}{\mathbb N}
\newcommand*{\Z}{\mathbb Z}
\newcommand*{\Q}{\mathbb Q}
\newcommand*{\R}{\mathbb R}
\newcommand*{\C}{\mathbb C}

\begin{document}
  \begin{frame}
    \titlepage
  \end{frame}

  \begin{frame}
    \frametitle{Unstetigkeitsbeweis mit dem Folgenkriterium}

    \uncover<+->{}
    \uncover<+->{Sei $f: \ldots$ eine Funktion mit $f(x)=\ldots$.}
    \uncover<+->{Diese Funktion ist unstetig an der Stelle $x_0=\ldots$.}
    \uncover<+->{Wählen wir nämlich die Folge $(x_n)_{n\in\N}$ mit $x_n=\ldots$, so liegen alle Folgenglieder im Definitionsbereich von $f$, und wir haben}

    \begin{align*}
      \uncover<+->{\lim_{n\to\infty} x_n = \ldots = x_0}
    \end{align*}

    \uncover<+->{Jedoch ist $\lim_{n\to\infty} f(x_n) \neq f(x_0)$.}
    \uncover<+->{Es ist nämlich}

    \begin{align*}
      \uncover<+->{\text{\emph{Beweis für }} \lim_{n\to\infty} f(x_n) \neq f(x_0)}
    \end{align*}
  \end{frame}

  \begin{frame}

  \end{frame}
\end{document}
